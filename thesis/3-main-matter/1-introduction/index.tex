\chapter{Introduction}
\label{chapter:introduction}

This chapter provides a summary of the content and organization of this thesis,
describing the context, motivations and high-level goals of this project and
how they are presented in this document.

\section{Content}

The ever-increasing availability of devices is creating an emerging class of
distributed systems, called collective adaptive systems, with application
domains such as smart cities, complex sensor networks and the \ac{IoT}
\cite{CAS-AggregateComputingBlocks}. The complexity of these systems calls for
new programming paradigms better suited for large-scale distributed systems,
such as aggregate computing \cite{FieldCalculus-AggregateComputing}.

Most state-of-the-art aggregate computing frameworks rely on a round-based
computation model, which is simple, but limited in terms of flexibility and
efficiency. To provide for the shortcomings of the round-based computation
model, new reactive approaches are currently in research, such as the FRASP
library \cite{FRASP}, which is the subject of this work.

FRASP provides a novel domain-specific language for combining the functional
reactive programming paradigm with the aggregate computing paradigm, extending
the former to be applied in distributed systems and in particular in collective
adaptive systems, while also extending the latter with a reactive computation
model, replacing the typical round-based one.

At the time of writing, FRASP is a research project and there are many ideas,
challenges and features still to be explored. However, the library requires a
consolidated test suite before further evolution, to assess the correctness of
its current implementation and prevent possible software regressions, that may
be due to unsuspected interactions between present and future features.

The main goal of this thesis is to implement a validated version of the FRASP
library, providing a clearer definition of its functionalities, verified
through adequate unit tests. Properties concerning FRASP programs will be
mainly evaluated via simulation, requiring a thorough analysis and validation
of the current simulator as well.

\section{Structure}

The content of the thesis will be presented in detail in the following
chapters.

First, Chapter \ref{chapter:background} provides an overview of the main
concepts and technologies used in this project, so that this document may be
self-contained.

\sidenote{TODO: describe other chapters...}

\section{Style}

The writing style adopted within this document provides intentional meaning to
the font styles used in words or sentences. Here follows a comprehensive list
of such font styles and their meanings:

\begin{itemize}
  \item \textit{Italic}: used to draw the reader's attention towards certain
        words or sentences.
  \item \textbf{Bold}: used to introduce a new concept that has never been
        mentioned before in the document.
  \item \texttt{Monospace}: used to reference an existing concept in the source
        code of the project.
\end{itemize}

\section{Prerequisites}

The following chapters may contain references to concepts related to
object-orient\-ed and functional programming, assuming that the reader is
familiar with such paradigms (specifically the Java \cite{Java} and Scala
\cite{Scala} documentations). Indeed, Scala has been adopted as the language of
choice in this document for abstracting over software interfaces and writing
pseudo-code, due to its clean and minimalistic functional syntax.
