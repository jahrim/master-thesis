% ! TeX root = ../../master-thesis.tex

\chapter{Introduction}
\label{chapter:introduction}

\section{Motivation and Goals}
\label{section:introduction:content}

The ever-increasing availability of devices is creating an emerging class of
distributed systems, called collective adaptive systems, with application
domains such as smart cities, complex sensor networks and the \ac{IoT}
\cite{CAS-AggregateComputingBlocks}. The complexity of these systems calls for
new programming paradigms better suited for large-scale distributed systems,
such as aggregate computing \cite{FieldCalculus-AggregateComputing}, whose
specifications directly describe robust collective behaviors for networks of
devices.

Most state-of-the-art aggregate computing frameworks rely on a round-based
computation model, which is simple, but limited in terms of flexibility and
efficiency. To provide for the shortcomings of the round-based computation
model, new reactive approaches are currently in research, such as the FRASP
(Functional Reactive Approach to Self-organization Programming) language
\cite{FRASP}, which is the subject of this work.

FRASP introduces a novel \ac{DSL} that combines the functional reactive
programming paradigm with the aggregate computing para\-digm. This extension
allows the application of the former in distributed systems, specifically in
collective adaptive systems. Additionally, it improves the latter with an
optimized reactive computation model, replacing the typical round-based one.

At the time of writing, FRASP is a research project and there are many ideas,
challenges, and features still to be explored. However, the language requires a
consolidated test suite before further evolution, to assess the correctness of
its current implementation, prevent possible software regressions, that may be
due to unsuspected interactions between present and future features, and
possibly discover unforeseen implications of the reactive model.

The main goal of this thesis is to implement a verified version of FRASP,
providing a clearer definition of its functionalities through adequate testing.
Properties concerning FRASP programs will be mainly evaluated via simulation,
requiring a thorough analysis and verification of the current simulator as
well.

\section{Structure}
\label{section:introduction:structure}

The content of this thesis will be presented in detail in the following
chapters. First, Chapter \ref{chapter:background} provides an overview of the
main concepts and technologies used in this project, so that this document may
be self-contained. Then, Chapter \ref{chapter:analysis} analyzes the objectives
and requirements of this thesis, defining an outline for the strategy to adopt.
Afterwards, Chapter \ref{chapter:design} describes the solution designed for
the project and Chapter \ref{chapter:implementation} delves into the details of
its concrete implementation. Towards the end, Chapter
\ref{chapter:verification} explains the verification methods applied to the
implemented solution, analyzing their results. Finally, Chapter
\ref{chapter:conclusions} provides a summary of the achievements and future
explorations of this project.

\section{Prerequisites}
\label{section:introduction:prerequisites}

The following chapters may contain references to concepts related to
object-orient\-ed and functional programming, assuming that the reader is
familiar with such paradigms (specifically the Java \cite{Java} and Scala
\cite{Scala} documentations). Indeed, Scala has been adopted as the language of
choice in this document for abstracting over software interfaces and writing
pseudocode, due to its clean and minimalistic functional syntax.

\section{Artifacts}
\label{section:introduction:artifacts}

The documented source code for this project is available in a public GitHub
repository\footnote{\url{https://github.com/jahrim/distributed-frp/tree/test/functional-test-suite}},
which can be downloaded to supplement the content of this document, providing
detailed insights into the implementation.