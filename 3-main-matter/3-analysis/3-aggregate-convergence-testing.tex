% ! TeX root = ../../master-thesis.tex

\section{Aggregate Convergence Testing}
\label{section:analysis:aggregate-convergence-testing}

The first step in consolidating the test suite is to implement several
aggregate \textbf{unit tests}, considering a FRASP construct as the
\textit{software unit}, and \textbf{integration tests}, involving FRASP
specifications (i.e., combinations of FRASP constructs). Best practices
\cite{UnitTesting} want unit tests to be:
\begin{itemize}
  \item \textit{Simple}: easy to implement. Indeed, inserting complex logic in
        a test requires such logic also to be tested, in order to ensure that
        the errors found by the test are not caused by faults in its logic.
        Moreover, simple tests can be easily understood, facilitating the
        detection of the cause of failure.
  \item \textit{Isolated}: independent of other unit tests (i.e., concerning
        a single software unit). Dependencies between unit tests make it more
        difficult to detect the cause of failure.
  \item \textit{Reproducible}: always yielding the same results under the same
        initial conditions (i.e., \textit{determinism}). Non-determinism may
        cause a test to succeed under breaking changes or to fail even with
        no changes at all.
  \item \textit{Finite}: yielding a result in a limited amount of time, ideally
        short for supporting frequent repeatability.
  \item \textit{Automated}: executed each time a relevant (preferably small)
        increment of software is completed.
\end{itemize}

By definition, integration tests cannot be isolated, however the other
properties should be preserved to the best of one's possibilities.

One of the challenges with aggregate tests is that the evolution of an
aggregate is naturally non-deterministic, due to the unpredictability of
communication in distributed systems. As a consequence, tests should be
carefully designed to reason about some deterministic higher-level behavior
exhibited by the underlying non-deterministic evolution of the aggregate (in
literature, \textbf{don't care non-determinism}), so that they can be
reproducible without forcing a deterministic evolution of the aggregate, which
would be unrealistic and reduce the importance of the tests.

In this sense, the primary strategy employed in this project is
\textbf{aggregate convergence tests}, which are based on the convergence of an
aggregate towards an expected stable state. Convergence is a property that can
be expressed in \textit{linear temporal logic} as $\lozenge \square P$
(\enquote{\textit{sometimes P will hold forever}}). In particular, it may be
interesting to evaluate the property $\lozenge \square \Psi_{expected}$,
understanding if the outputs of an aggregate will eventually reach and hold the
expectation $\Psi_{expected}$. However, convergence can only be verified for
self-stabilizing specifications (e.g., non-oscillating), assuming a finite
number of changes in the environment.
