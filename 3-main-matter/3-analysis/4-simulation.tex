% ! TeX root = ../../master-thesis.tex

\section{Simulation}
\label{section:analysis:simulation}

Since a simulator is already provided within FRASP, simulation will be used as
a formal verification method for properties in aggregate tests. However, the
current simulator is basic and lacks some properties required for adequate
testing (referring to the previous Section
\ref{section:analysis:aggregate-convergence-testing}). In particular, an
adequate simulator for aggregate tests should enjoy the following properties:
\begin{itemize}
  \item \textit{observability}: during a simulation, it should be possible for
        external entities to reconstruct the state of the simulation from its
        outputs. For aggregate tests, a simulation should expose at least the
        state of the aggregate and the progress of the simulation. Additionally,
        it would be useful to have access to individual, regional and global
        views of the aggregate. At the moment, the simulator does not expose
        any outputs to other entities, instead, it only shows the outputs of
        individual devices to the user through a console.

        This property is required for automated aggregate tests.
  \item \textit{controllability}: it should be possible to control a
        simulation, leading it to a stable state when its execution does not
        converge in a finite amount of time. For aggregate tests, a simulation
        should at least have the capability to be halted. At the moment, a
        simulation starts as the simulator is created and continues
        indefinitely, forever reacting to the next event.

        This property is required for finite aggregate tests.
  \item \textit{fairness}: in aggregate tests, it should always be possible for
        every device to compute an export in the future. At the moment, the
        simulator relies on the scheduling of the underlying runtime to achieve
        fairness.

        This property is required to support self-stabilisation.
  \item \textit{efficiency}: it should be optimized to minimize execution time,
        possibly leveraging parallelism. At the moment, the simulator supports
        concurrent execution, but it suffers from critical races (and other
        deeper problems discussed later in Section
        \ref{section:design:concurrent-simulation}).

        This property is required to reduce the computational costs of testing
        and support frequent repeatability.
  \item \textit{reproducibility}: multiple simulations should yield similar
        results under similar initial configurations, implying the ability to
        execute a simulation under similar conditions multiple times
        (\textit{repeatability}) or under different conditions
        (\textit{replicability}).

        Naturally, this property is required for reproducible tests.
\end{itemize}

Since the current simulator does not enjoy all the aforementioned properties,
an extension of the simulator is due. Most importantly, observability and
controllability should be provided for testing. However, in doing so, one
should be mindful of preserving the reactive execution model of FRASP as is.
Indeed, a problem with testing through simulation is that the results of the
tests may be influenced by the implementation of the simulator.
