% ! TeX root = ../../master-thesis.tex
\section{Aggregate Testing}
\label{section:analysis:aggregate-testing}

An aggregate test should verify that an aggregate behaves as expected with
respect to a given specification. However, while its purpose may be clear at an
abstract level, a more detailed analysis is required to determine its concrete
implications (e.g., what does it mean for an aggregate to behave as expected?).

Leveraging the network operational semantics of field calculus
\cite{CAS-AggregateComputingBlocks}, the evolution of an aggregate can be
described by a transition system in which each transition is $N_{t}
\xrightarrow{act} N_{t+1}$ with:

\begin{align*}
   & N_t::=(\Psi_t, E_t)                & : & \textrm{ the state of the aggregate at time t}                                 & \\
   & E_t::=(\tau_t, \Sigma_t)           & : & \textrm{ the state of the environment at time t}                               & \\
   & \Psi_t                             & : & \textrm{ the output of all the devices at time t}                              & \\
   & \tau_t                             & : & \textrm{ the topology of the network at time t}                                & \\
   & \Sigma_t                           & : & \textrm{ the percepts of the sensors at time t}                                & \\
   & act::=\delta_{k} \textrm{ or } env & : & \textrm{ a change in the aggregate}                                            & \\
   & \delta_{k}                         & : & \textrm{ a change due to the } k^{th} \textrm{ device broadcasting its export} & \\
   & env                                & : & \textrm{ a change in the environment}                                          & \\
\end{align*}

In a \textit{static} environment $E_0$, transitions can be reduced to the form
$\Psi_{t} \xrightarrow{\delta_{k}} \Psi_{t+1}$, i.e., the transition system is
uniquely described by an initial aggregate state and the sequence of all the
device exports.

Once the evolution of an aggregate is expressed as a transition system, formal
verification techniques for transition systems may be applied to aggregates as
well. In particular, one can express properties on aggregates using
\textit{propositional logic}, for static attributes, or even \textit{temporal
logics}, for dynamic or branching attributes. Then, properties may be verified
through formal techniques such as \textit{model checking} or
\textit{simulation}.

Properties are used to formally define the expectations for the evolution of an
aggregate, including the output of the devices in the network, their percepts,
the topology of the network, environmental changes and device communication.
Expectations may involve one, some or all of the devices in the network,
therefore properties can be:
\begin{itemize}
  \item \textit{global}: a property of the whole aggregate (e.g., \texttt{mid}
        should evaluate to the identifiers of all the devices in the network).
  \item \textit{regional}: a property of a group of devices in an aggregate
        (e.g., \texttt{branch (isRed)\{obstacle\}\{???\}} should evaluate to
        \texttt{obstacle} for all red devices in the network).
  \item \textit{individual}: a property of a single device in an aggregate
        (e.g., device 0 should always be a source of potential).
\end{itemize}
