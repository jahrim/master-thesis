% ! TeX root = ../../master-thesis.tex

\chapter{Conclusions}
\label{chapter:conclusions}

This thesis started with the goal of providing proper verification for the
functionalities of FRASP; explored several strategies for observing and
controlling the reactive execution of aggregate specifications, leading to a
step-by-step reactive execution model; defined a concrete solution for
evaluating convergence-based spatio-temporal properties of FRASP systems
through simulation; and developed an extensive test suite, providing valuable
insights on the current state of the library and future challenges to overcome.

The results proved the overall soundness of FRASP, consolidating its
foundations in support of upcoming improvements and extensions. Some issues
were identified and analysed, including a limitation in referencing the
computation of aggregate specifications within other specifications, and a
scheduling problem with the previous and current simulators, causing
inconsistencies during the evolution of aggregates over time.

The contributions of this project include a modular implementation of several
simulators, a collection of operators for the analysis and monitoring of
reactive variables, and explicit support for dynamic environments and proper
sensors. Unfortunately, while the solution was designed for concurrency, the
strong consistency of the underlying reactive engine proved to negate
parallelism, limiting the benefits of concurrency. On the one hand, there is
opportunity for reducing the complexity of the current design by giving up
concurrency, possibly embracing a complete \ac{FRP} implementation without
compromises; on the other hand, some questions arise about the implications of
such constraints on the deployment of aggregate specifications in real-world
distributed systems, specifically collective adaptive systems, hinting towards
the necessity for further research into distributed reactive solutions.
