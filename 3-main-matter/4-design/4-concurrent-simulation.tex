% ! TeX root = ../../master-thesis.tex

\section{Concurrent Simulation}
\label{section:design:concurrent-simulation}

Concurrency in reactive simulations can be achieved by delegating the
propagation of change to some workers (e.g., a thread pool). In particular, in
Sodium, concurrency can be achieved by removing a dependency between a consumer
and a producer in a computational graph, then listening to the events of the
producer and delegating to a worker the propagation of each event towards the
consumer. However, this approach is limited to concurrency and cannot achieve
parallelism, due to Sodium's transactional system.

As discussed in Section \ref{section:background:technologies:sodium}, Sodium's
transactions are executed one at a time to guarantee glitch freedom, trading
off parallelism to ensure consistency. Later, it was discovered that
transactions are executed sequentially even among independent computational
graphs. Therefore, unless the computation of a device is detached from the
computational graph (i.e., executed outside the FRP engine), concurrent
simulations cannot be executed in parallel, in fact concurrent events are still
processed sequentially.

Moreover, the deployment of the reactive execution model in real distributed
systems is still unclear, possibly hinting towards the exploration of
distributed reactive programming solutions. Further research could discover the
effects of Sodium's consistency in the evolution of aggregate of devices and
evaluating the possibility of achieving the same level of consistency in
large-scale distributed systems, such as \ac{CAS}s.
