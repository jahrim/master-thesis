% ! TeX root = ../../../master-thesis.tex

\subsection{Sodium}
\label{section:background:technologies:sodium}

\textbf{Sodium}\footnote{Repository at: \url{https://github.com/SodiumFRP}} is a
BSD-licensed library implementing \ac{FRP} in several languages (including
Java), inspired by many previous implementations of \ac{FRP}. Sodium
is meant to be a \textit{true} \ac{FRP} implementation, in the sense that it
provides full compositionality compared to other implementations (e.g., Reactive
Extensions (Rx)\footnote{Repository at: \url{https://github.com/ReactiveX}},
which is not glitch-free) \cite{FRP}.

In Sodium, behaviours are modelled as \textbf{cells} with denotation
\texttt{Cell[V]}, indicating a time-varying value of type \texttt{V}, while
event streams are modelled as simply \textbf{streams} with denotation
\texttt{Stream[E]}, indicating a sequence of emissions of events of type
\texttt{E}. In particular, a \texttt{Stream} is defined as a list of events
bound to the time when they were fired, while a \texttt{Cell} is defined as a
pair of its initial value together with a \texttt{Stream} of its updates over
time.

Time is represented as a sequence of \textbf{transactions}, which can be
interpreted as atomic time units. Only one transaction at a time can be
executed by the engine of Sodium, even when considering multiple independent
computational graphs. During a transaction, first all events are processed
simultaneously keeping all values constant (i.e., immutable
\textbf{transactional context}), then all time-varying values are updated
accordingly. A transaction is started automatically each time an event is
pushed in the computational graph and closed only after its corresponding
propagation of change has been completed. Alternatively, it is possible to
create a new transaction explicitly using the \texttt{Transaction.run} method
(e.g., useful for sending simultaneous events, graph initialization or handling
forward references).

Sodium provides a set of built-in core primitives for building static acyclic
computational graphs (Listing \ref{listing:sodium-basic}), creating and
combining \texttt{Cell}s and \texttt{Stream}s. These include:

\begin{itemize}
  \item \texttt{never}: create a new \texttt{Stream} that will never emit
        events.
  \item \texttt{map}: given a \texttt{Stream} $s$ in input, create a new
        \texttt{Stream} $s'$ whose events are the events of $s$ transformed
        with a given \texttt{mapping} function. An analogous operation is
        provided for \texttt{Cell}s.
  \item \texttt{filter}: given a \texttt{Stream} $s$ in input, create a new
        \texttt{Stream} $s'$ whose events are the events of $s$, discarding
        those which do not satisfy a given \texttt{predicate}.
  \item \texttt{merge}: given two \texttt{Stream}s $s_1$ and $s_2$, create a
        new \texttt{Stream} $s'$ whose events are the events fired by either
        $s_1$ or $s_2$, combining their simultaneous events with a given
        \texttt{merging} function.
  \item \texttt{snapshot}: given a \texttt{Stream} $s$ and a \texttt{Cell} $c$,
        create a new \texttt{Stream} $s'$ whose events are the events of $s$
        combined with the most recent value of $c$ using a given
        \texttt{combine} function.
  \item \texttt{constant}: create a \texttt{Cell} $c$ holding a given
        \texttt{value} forever.
  \item \texttt{hold}: given a \texttt{Stream} $s$, create a \texttt{Cell} $c$
        holding a given \texttt{initial} value, which is updated each time $s$
        fires a new event. In particular, $s$ is the \texttt{Stream} of updates
        of $c$.
  \item \texttt{sample}: given a \texttt{Cell} $c$, obtain its most recent
        value. This primitive should not be used when mapping or lifting
        \texttt{Cell}s as it would break referential transparence, which is
        preserved for other primitives by exploiting the immutability of
        transactional contexts.
  \item \texttt{lift}: given two \texttt{Cell}s $c_1$ and $c_2$, create a new
        \texttt{Cell} $c$ whose value is obtained by combining the values of
        $c_1$ and $c_2$ using a given \texttt{operator}. In particular,
        the value of $c$ is updated each time the values of $c_1$ or $c_2$ are
        updated. Note that lifting is explicit in Sodium.
\end{itemize}

\lstinputlisting[
  language=Scala,
  caption={An abstract view on the Sodium primitives for constructing static
      acyclic computational graphs. Some primitives can be derived as a
      combination of the others (e.g., \texttt{constant} and
      \texttt{snapshot}).},
  captionpos=b,
  label={listing:sodium-basic}
]{resources/listings/sodium-basic.txt}

Sodium also provides support for dynamic computational graphs, including graph
expansion, reduction and more general sub-graph substitution. A dynamic
computational graph can be represented as a time-varying computational graph,
that is a \texttt{Cell} holding a reactive variable as value (either other
\texttt{Cell}s or \texttt{Stream}s). In particular, two switching operators are
implemented in Sodium (Listing \ref{listing:sodium-switching}):
\texttt{switchS} builds a dynamic computational graph from a \texttt{Cell} of
\texttt{Stream}s $cs$, creating a new \texttt{Stream} $s'$ whose events are the
events of the most recent \texttt{Stream} held by $cs$; \texttt{switchC} works
similarly for \texttt{Cell} of \texttt{Cell}s.

\lstinputlisting[
  language=Scala,
  caption={An abstract view on the Sodium primitives for constructing dynamic
      computational graphs.},
  captionpos=b,
  label={listing:sodium-switching}
]{resources/listings/sodium-switching.txt}

Support is also provided for cyclic computational graphs. However, since a node
declares its dependencies on other defined nodes during its creation, cyclic
dependencies are not possible without a mechanism for forward referencing,
allowing a node to declare a dependency on another node that is yet to be
defined (e.g., itself). Sodium allows forward referencing in Java by decoupling
the declaration and definition of a node using the type \texttt{CellLoop[V]}
(or \texttt{StreamLoop[E]}), which is used for declaring a node that will be
assigned later to a defined \texttt{Cell} (or \texttt{Stream}) through its
method \texttt{loop}. In other words, \texttt{CellLoop} acts as a placeholder,
referencing a \texttt{Cell} that is not yet available. Still, declaration and
definition should happen conceptually at the same time to avoid the propagation
of change to empty references, hence a \texttt{CellLoop} must be declared and
assigned within the same transaction.

\lstinputlisting[
  language=Scala,
  caption={An abstract view on the Sodium primitives for constructing cyclic
      computational graphs.},
  captionpos=b,
  label={listing:sodium-looping}
]{resources/listings/sodium-looping.txt}

Interoperability with non-\ac{FRP} software interfaces is provided via a set of
\textbf{operational primitives} (Listing \ref{listing:sodium-operational}),
which are excluded from the core primitives, since their incorrect usage may
break some properties of Sodium. A broker between a \ac{FRP} interface and a
non-\ac{FRP} interface can be implemented using the type \texttt{CellSink[V]}
(or \texttt{StreamSink[E]}), which is a \texttt{Cell} (or \texttt{Stream}) that
supports event pushing. In particular, the \texttt{send} primitive implements
non-FRP to FRP interactions, allowing pushing an update to a \texttt{CellSink}
and managing the propagation of change through a push-based evaluation model
(i.e., the caller of \texttt{send} will update all the dependent nodes in the
computation graph). Conversely, the \texttt{listen} primitive implements FRP to
non-FRP interactions, allowing the registration of a callback to execute any
time the state of a \texttt{Cell} is updated (such subscription can be
cancelled using the returned \texttt{Listener}). Note that using \texttt{send}
within a callback is not allowed, as it could be used to implement custom
primitives that violate compositionality. For the same reasons, Sodium
discourages and forbids inheritance of its types. Instead, custom primitives
should be implemented as a combination of the core primitives to preserve
compositionality.

\lstinputlisting[
  language=Scala,
  caption={An abstract view on the Sodium operational primitives.},
  captionpos=b,
  label={listing:sodium-operational}
]{resources/listings/sodium-operational.txt}

Additionally, Sodium offers other operational operators to tackle some specific
practical problems (e.g., \texttt{value}, \texttt{updates}, \texttt{split},
\texttt{defer}\dots) and many more helper primitives to facilitate the
construction of computational graphs (e.g., \texttt{accum}, \texttt{collect},
\texttt{sequence}, \texttt{gate}\dots). While these operators won't be
discussed here, since they are not as relevant for this project, more
information about them and Sodium can be found in the book \cite{FRP}. The book
also describes some helpful \ac{FRP} patterns, such as the \textbf{calming}
pattern, useful to create \textbf{calm} reactive variables, which avoid firing
consecutive repetitions of the same event, reducing redundant re-computations.

The authors compare other standard event-programming paradigms (specifically
the observer pattern) to Sodium, highlighting several bugs that are common in
the former, which are banished in the latter if used as intended. In
particular, Sodium promises to solve the following problems:
\begin{itemize}
  \item \textit{Unpredictable order}: in complex networks of callbacks, it is
        difficult to track the order in which they are executed. Sodium
        abstracts over event ordering making it completely undetectable.
  \item \textit{Missed first event}: it is difficult to guarantee that
        callbacks are registered before the first event. Sodium can solve the
        problem by initializing the program within a transaction.
  \item \textit{Messy state}: callbacks tend to describe behaviours as state
        machines, which are difficult to maintain. Sodium solves the problem
        using the declarativity of the \ac{FRP} paradigm.
  \item \textit{Threading issues}: executing callbacks in parallel may lead to
        deadlock due to synchronization. Sodium solves the problem by executing
        only one transaction at a time.
  \item \textit{Leaking callbacks}: forgetting to deregister a callback from
        a producer causes memory leaks and wastes CPU time. Sodium
        automatically deregisters callbacks that are not used any longer.
  \item \textit{Accidental recursion}: it is easy to introduce accidental
        cyclic dependencies between nested callbacks. Sodium solves the problem
        using the declarativity of the \ac{FRP} paradigm.
\end{itemize}
In addition, Sodium grants the compositionality required to tackle the growing
complexity of scalable systems.
