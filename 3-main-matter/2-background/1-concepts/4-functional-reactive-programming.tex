% ! TeX root = ../../../master-thesis.tex

\subsection{Functional Reactive Programming}
\label{section:background:concepts:frp}

\textbf{\ac{FRP}} is a subset of both \textbf{reactive programming} and
\textbf{functional programming}, retaining the advantages of reactive
programming, while promoting \textbf{compositionality}, which is a
property of semantics, holding if the meaning of an expression is solely
determined by the meaning of its parts and the rules used to combine them
\cite{FRP}.

In functional programming, compositionality is achieved by expressing software
behaviours as \textbf{pure functions}, that is functions in the mathematical
sense of the term. Pure functions produce no observable side effects when
applied and are \textbf{referentially transparent}, meaning that different
applications of a function to the same input always produce the same outputs.
To attain referential transparence, functions should avoid referencing
\textit{shared mutable data}, so that their behaviour is kept constant since
their definition and has no side effects.

In reactive programming, compositionality also requires glitch freedom, as
observable glitches may invalidate the behaviour expressed by a function (e.g.,
the behaviour of a function may change due to inconsistent handling of
simultaneous events by the underlying evaluation model).

Compositionality is essential for dealing with complex software, tackling its
complexity by composition of simpler components that are easier to reason
about. Moreover, it deals with scalable software, tackling their growing
complexity over time by facilitating the addition of new features to existing
composable applications.
