% ! TeX root = ../../../master-thesis.tex

\subsection{Collective Adaptive Systems}
\label{section:background:concepts:cas}

\textbf{Collective systems} are distributed situated systems composed of a
potentially large set of computing components, that are competing or cooperating
to achieve a specific goal, interacting with each other and adapting to the
changes of their environment \cite{CAS}.

The behaviour of a collective system as a whole is an expression of
\textbf{collective intelligence} or \textbf{swarm intelligence}, in fact it
emerges from the behaviours of its individual components, the local
interactions between them and with their environment \cite{SwarmIntelligence}.
These concepts come from the study of self-organising groups of entities in
nature (e.g., ant colonies, bird flocks) applied to computer science, in
pursuit of \textbf{adaptiveness} and in particular \textbf{self-organisation}.

Adaptiveness is the ability of a system to change its behaviour depending on
the circumstances to better achieve its goals and it is so important in the
applications of collective systems that they are often called directly
\textbf{\acp{CAS}}. In fact, adaptiveness grants collective systems with the
robustness needed to address unforeseen changes in operating conditions, which
are typical in real-world environments (e.g., network failures, open networks,
mobile components).

General adaptiveness can be obtained using different strategies, including
centralised approaches in which a designated control system changes the
behaviours of the system components depending on their perception of the local
environment. However, \acp{CAS} achieve adaptiveness specifically through a
decentralised approach called self-organisation, in which complex global
ordered structures (e.g., collective behaviours) form as a consequence of
simple local seemingly-chaotic interactions (e.g., local communication,
stigmergy) \cite{SelfOrganization}. This kind of adaptiveness is also called
\textbf{self-adaptiveness}, as it arises from the system itself without any
external contributor.

Collective systems are especially complex, in fact in collective intelligence
the connection between the individual behaviours of the system components and
the collective behaviour of the system is rarely straightforward. As a
consequence, it may be difficult to design the individual components starting
from the goal that the collective system should achieve. To tackle such
complexity, one should adopt stricter and more formal approaches to software
engineering, such as anticipating the verification of the system already during
its design, using formal verification techniques such as \textbf{model
checking} and \textbf{simulation}.

The applications of collective systems concern domains such as smart cities,
complex sensor networks and the \ac{IoT} \cite{CAS-AggregateComputingBlocks},
including pedestrian navigation (e.g., crowd evacuation), collective motion
(e.g., drone fleet control \cite{CollectiveMotion-Quadcopter}) and pervasive
\ac{IoT}.