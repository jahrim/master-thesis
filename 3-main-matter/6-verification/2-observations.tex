% ! TeX root = ../../master-thesis.tex

\section{Observations}
\label{section:verification:observations}

The test suite yielded positive results, overall proving the FRASP library to
be working as expected. However, the tests also confirmed a couple of issues,
described in the following sections.

\subsection{Referential Transparence}

The aggregate computing \ac{DSL} in FRASP enjoys referential transparence,
granting compositionality for its constructs. As a consequence of referential
transparence, the same behaviour can be defined by replacing the values of a
program with the constructs producing those values, like in the following
example.

\lstinputlisting[language=Scala, nolol=true]{resources/listings/branch-observation.txt}

The above listing defines three specifications: \texttt{collectNeighbors}, for
collecting the set of neighbours of all the devices (line 1);
\texttt{branchAndCollectNeighbors}, for partitioning the network and then
collecting the set of neighbours in each partition (line 4); finally,
\texttt{collectNeighborsAndBranch}, for collecting the set of neighbours of all
devices and then partitioning the network (line 7). Unexpectedly,
\texttt{branchAndCollectNeighbors} and \texttt{collectNeighborsAndBranch} yield
the same results, due to referential transparence (line 5 is semantically
equivalent to lines 8-9). In summary, \texttt{collectNeighborsAndBranch} does
not work as intended.

While referential transparence is not at all a negative property, FRASP
currently has no mechanisms to reference the results of other specifications
within the local scopes of its constructs. In the example, the forks of a
\texttt{branch} construct cannot access the set of neighbours collected before
partitioning. This was a known issue\footnote{
\url{https://github.com/cric96/distributed-frp/issues/1}}, confirmed by the
test suite of FRASP.

\subsection{Inconsistency of Loop}

The test suite found some specifications based on the \texttt{loop} construct
producing inconsistent results. The inconsistencies manifested as an infinite
execution for non-trivial self-stabilizing specifications, expected to
stabilize in a finite amount of time. After further analysis, it was discovered
that the inconsistent specifications had in common an evolution over time
dependent both on the previous value $P$ and at least one external source of
change $S$ (e.g., the \texttt{nbr} or \texttt{sensor} constructs).

The cause of the problem is attributed to the fact that the previous and
current simulators do not propagate exports immediately, but their transmission
is scheduled for later in the future. However, the implementation of the
\texttt{loop} construct relies on self-communication for evaluating the
previous value $P$. As a consequence, there is a delay between the generation
of the new value of \texttt{loop} and the corresponding update of $P$.

During the delay, other dependencies of the \texttt{loop} construct, such as
$S$, may trigger a propagation of change, generating a new inconsistent value
of \texttt{loop}, based on the stale value of $P$ and the new value of $S$. The
problem is aggravated by the fact that the inconsistent value is also scheduled
for transmission, including self-communication. As a result, any such
inconsistent value starts an inconsistent evolution over time, concurrently
with the others. This phenomenon can quickly build up, causing the simulation
to never cease firing new events.

Analysing this issue was especially difficult, since most affected
specifications in the test suite, including the gradient, achieve a robust
convergence even with the inconsistent values, probably due to the type of
operations involved in the specification (e.g., accumulating information by
evaluating the maximum value).

An example of highly susceptible specification is the following.

\lstinputlisting[language=Scala, nolol=true]{resources/listings/zip-with-round-observation.txt}

The specification in the example, namely \texttt{zipWithRound}, simply creates
a \texttt{Flow} whose exports are the exports of the input \texttt{Flow}, bound
to the time when they are transmitted (\textit{round}), ignoring repeated
consecutive exports. However, if the input \texttt{Flow} has an external
dependency (e.g., a \texttt{nbr} construct), the output \texttt{Flow} will
likely emit inconsistent values. This specification nicely demonstrates the
issue, as inconsistent values are manifested as repeated or even oscillating
rounds in the exports.

A solution to this issue could be to implement self-communication as an
instantaneous transmission, however, a concrete implementation is yet to be
developed.
