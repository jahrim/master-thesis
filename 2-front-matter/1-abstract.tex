% ! TeX root = ../master-thesis.tex

The growing relevance of large-scale distributed systems, including collective
adaptive systems, has inspired the research for novel aggregate computing
paradigms, addressing the complexity of programming macro-level behaviours over
sizeable networks of devices. One of the most recent advancements in this
direction is the development of a reactive execution model for such systems,
embodied in the FRASP library, which overcomes several limitations of the
previous round-based execution models, such redundant re-computations.

The objective of this thesis is to consolidate FRASP by developing an extensive
test suite, demonstrating the correctness of the current implementation,
supporting future extensions of the library, and providing valuable insights
into the implications of the reactive execution model and the challenges to
overcome. With this goal in mind, the event-driven simulator provided by FRASP
has been re-designed to enable the evaluation of spatio-temporal properties on
the evolution of aggregate systems, focusing on the convergence of
self-stabilizing specifications towards an expected stable state.

In conclusion, the test suite verified the overall correctness of the FRASP
library, except for a few problems regarding the implementation, which have
been identified and analysed. Additionally, the strong consistency provided by
the underlying reactive engine raised concerns about deployment in large-scale
distributed systems, to be investigated in future research.